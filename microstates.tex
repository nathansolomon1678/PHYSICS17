\documentclass[12pt]{article}
\usepackage[margin=1in]{geometry}
\usepackage{amsmath}
\usepackage{amsfonts}
\usepackage{tikz-cd}

\begin{document}

\title{Microstates}
\author{Nathan Solomon}
\maketitle

\section{Macrostates vs. microstates}
Suppose you have $N$ particles which each have a nonnegative integer amount of energy, and the total energy of those particles is $E$, where $E \geq N$. In this example, suppose $N=6$ and $E=8$. Each way of distributing the energy between $N$ unordered particles is a macrostate, and each way of distributing the energy between $N$ ordered particles is a microstate. Counting microstates is useful because every microstate is equally likely, so we can figure out the probability a that particle will be at a given energy level. In this example, there are 20 macrostates and 1287 microstates.

Each microstate can be represented by a "stars-and-bars" diagram, where each star is a unit of energy. For example, if one particles has 4 energy, 2 particles have 2 energy, and the rest have zero energy, the diagram could look like
\[\star \star \star \star | \star \star | \star \star | | |\]
But by convention, each partition in a stars-and-bars diagram must have at least one star, so we add one star to each partition by default, and instead write
\[\star \star \star \star \star | \star \star \star | \star \star \star | \star | \star | \star\]
This diagram has the same distribution of energies, but in a different order, meaning it represents the same macrostate, but a different microstate.
\[\star \star \star | \star \star \star \star \star | \star \star \star | \star | \star | \star\]
That microstate can be permuted $\frac{6!}{2!3!} = 60$ different ways, meaning that that macrostate contains 60 microstates.

\section{Nasty combinatorics stuff}

\[\Omega(E, N) := \text{\# of microstates} = \sum_{\text{macrostates with N particles \& energy E}} (\text{\# of permutations of that macrostate})\]

Going back to the first diagram (the one where some partitions have no stars), we have a total of $N+E-1$ characters: $E$ stars and $N-1$ bars. So the number of ways to arrange those characters is
\[\Omega(E, N) = {{N+E-1} \choose E} = {{N+E-1} \choose {N-1}} = \frac{(N+E-1)!}{E!(N-1)!} \]

In the example where $E=8$ and $N=6$, that tells us that there are $\Omega(8, 6) = 1287$ microstates.

\section{Deriving the discrete probability distribution}

Define the function
\begin{align*}
    f(e, E, N) :&= \sum_{\text{microstates with N particles \& total energy E}} (\text{\# of particles with energy }e) \\
    &= \sum_{n=1}^{\lfloor E / e \rfloor} n \cdot (\text{\# of microstates where exactly n particles have energy e})
\end{align*}

Since each microstate has $N$ particles, we know that
\[\sum_{e=0}^E f(e, E, N) = N \cdot \Omega(E, N)\]

We also know that
\[f(E, E, N) = N\]
and that

???

\[f(e, E, N) = N {{E + N - e - 2} \choose {N - 2}} = N {{E + N - e - 2} \choose {E - e}}\]

I HAVE ABSOLUTELY NO IDEA HOW TO GET THAT FORMULA BUT I THINK IT MIGHT WORK???

Now we can calculate the probability $p(e, E, N)$ of finding a particle with energy $e$ (given that there are $N$ particles whose total energy is $E$). Note that the average number of particles in each microstate with energy $e$ is equal to $N$ times $p(e, E, N)$.
\[p(e, E, N) := \frac{f(e, E, N)}{N \Omega(E, N)} = \frac{(E+N-e-2)!}{(E-e)!(N-2)!} \cdot \frac{E!(N-1)!}{(N+E-1)!} = \frac{N-1}{E+N-1-e} \cdot \prod_{i=1}^e \frac{E+1-i}{E+N-i} \]

\end{document}
