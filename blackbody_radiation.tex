\documentclass[12pt]{article}
\usepackage[margin=1in]{geometry}
\usepackage{amsmath}
\usepackage{amsfonts}
\usepackage{esint}
\usepackage{tikz-cd}

\begin{document}

\title{Blackbody Radiation}
\author{Nathan Solomon}
\maketitle

\section{Density of states in $k$-space}
For a standing harmonic wave in a one-dimensional container of length $L$, the ends must be nodes, so the wavevector $k$ can be any integer multiple of $\pi/L$. Each harmonic wave that satisfies the boundary conditions represents one possible state, so in $k$-space, the density of possible states is $L/\pi$. However, a standing sinusoidal wave with wavevector $k$ is exactly the same as a standing wave with wavevector $-k$, so to avoid counting the same state twice, we assume $k$ is positive. Now if you consider standing waves in a 3-dimensional box with volume $V$, and let $k$ be the 3D wavevector, then the density of states in $k$-space is $V/\pi^3$. That only applies to the quandrant where all 3 components of $k$ are positive, since the density of states is zero elsewhere. To make the next step easier, we can pretend the density of states in $k$-space is $(L/2\pi)^3$ everywhere. Then if we instead use $k$ to mean the wavenumber instead of the wavevector, then we can integrate over cencentric spherical shells in $k$-space to find that the density of states is
\[ dN = (4 \pi k^2 dk) \left( \frac{L}{2 \pi} \right)^3 = \frac{V k^2}{2 \pi^2} dk \]

But if we're these are light waves, there are two polarization states for each axis (this is sort of like the degeneracy of states), we double that to get
\[ dN = \frac{V k^2}{\pi^2} dk \]

Since we're considering photons, we can use the substitution $k = 2 \pi f / c$ where $f$ is temporal frequency. That also means that $dk = 2 \pi df / c$, so the density of states as a function of $f$ is
\[ dN = \frac{8 \pi V f^2}{c^3} df \]

\section{Spectral energy density}
The spectral energy density, $u(f, T)$ represents the energy per volume per frequency. That's equal to the energy per state, times the density of states, times the probability of being in each of those states, divided by the volume. Since the energy is $E = hf$, that works out to
\[E(f) \cdot N(f) \cdot \frac{1}{e^{E(f)/k_B T} - 1} \cdot \frac{1}{V} \]
Plugging in the formulas for $E$ and $N$, that becomes
\[u(f, T) = \frac{8 \pi h f^3}{c^3 (e^{hf/k_B T} - 1)} df\]
If you substitute in $f = c / \lambda$ and $df = - c d\lambda / \lambda^2$, you can also get the spectral energy density in terms of wavelength (although you also have to take the absolute value of that, since increasing $f$ means decreasing $\lambda$)
\[ u(\lambda, T) = - \frac{8 \pi h}{\lambda^3 (e^{h c / \lambda k_B T} - 1)} \Bigg( - \frac{c \, d\lambda}{\lambda^2} \Bigg) = \frac{8 \pi h c}{\lambda^5 (e^{h c / \lambda k_B T} - 1)} d\lambda \]
We also sometimes use the spectral radiance, $B$, instead of spectral energy density. Spectral radiance is the power emitted per unit area.
\[ B := \frac{c u}{4 \pi} \]

At large frequencies and small temperatures, you can approximate the Bose-Einstein distribution with the Maxwell-Boltzmann distribution:
\[ \lim_{f/T \rightarrow \infty} u(f, T) = \frac{8 \pi h f^3}{c^3} e^{-hf/k_B T} df \]
That approximation is called Wien's law, Wien's distribution law, or Wien's approximation. Don't confuse it with Wien's displacement law, which is described below. Similarly, you can come up with an estimation for small frequencies and large temperatures by using a first-order Maclauring series expansion of the Boltzmann factor:
\[ \lim_{T/f \rightarrow \infty} u(f, T) = \frac{8 \pi h f^3}{c^3 ((1 + hf/k_B T) - 1)} df  = \frac{8 \pi f^2 k_B T}{c^3} df \]
That equation is the classical approximation, called the Rayleigh-Jeans law. For a long time scientists were bamboozled by the ``ultraviolet catastrophe" -- the idea that according to this law, $u(f, T)$ should go to infinity as you look at higher and higher frequencies, which is clearly incorrect. This was solved by the correct formulas for $B$ and $u$ derived above, which are called Planck's law.

\section{Wien's displacement law}
Let $\lambda_{\text{max}}$ be the value of $\lambda$ such that $u(\lambda, T)$ is maximized. Note that $\lambda_\text{max}$ is a function of temperature. Wien's displacement law (different from Wien's law) states that $\lambda_\text{max} T$ is a universal constant.

You can also define $f_\text{max}$ to be the value of $f$ such that $u(f, T)$ is maximized. By the relation $\lambda = c / f$, that means $c T / f_\text{max}$ is also a constant, BUT NOT THE SAME CONSTANT. In fact, the latter is slightly larger.

\section{Stefan's law}
Imagine a hole in a box. The rate of effusion of particles is $A c / 4$, so the power emitted from that blackbody is
\begin{align*}
    P &= \frac{A c}{4} \int_0^\infty u(f, T) df \\
      &= \frac{A c}{4} \int_0^\infty \frac{8 \pi h f^3}{c^3 (e^{hf/k_B T} - 1)}  df \\
      &= \frac{2 \pi A h}{c^2} \int_0^\infty \frac{f^3}{e^{hf/k_B T} - 1} df \\
      &= \frac{2 \pi A h}{c^2} \Bigg( \frac{k_B T}{h} \Bigg)^4 \int_0^\infty \frac{x^3}{e^x - 1} dx \\
      &= \frac{2 \pi k_B^4 T^4 A}{c^2 h^3} \int_0^\infty \frac{x^3}{e^x - 1} dx
\end{align*}
That funky integral is difficult to evaluate. Lucikly online resources have figured out how to do it for us.
\begin{align*}
    \int_0^\infty \frac{x^3}{e^x - 1} dx &= \int_0^\infty \frac{x^3 e^{-x}}{1 - e^{-x}} dx \\
    &= \int_0^\infty x^3 e^{-x} \sum_{n=0}^\infty (e^{-x})^n dx \\
    &= \int_0^\infty x^3 \sum_{n=1}^\infty e^{-nx} dx \\
    &= \sum_{n=1}^\infty \int_0^\infty x^3 e^{-nx} dx \\
    &= \sum_{n=1}^\infty \int_0^\infty \left( \frac{y}{n} \right)^3 e^{-y} \left( \frac{dy}{n} \right) \\
    &= \sum_{n=1}^\infty \frac{1}{n^4} \cdot \int_0^\infty y^3 e^{-y} dy \\
    &= \zeta(4) \cdot \Gamma(4) \\
    &= \frac{\pi^4}{90} \cdot (3!) \\
    &= \frac{\pi^4}{15}
\end{align*}
Therefore we can solve for the ``blackbody radiant emittance" ($j^\star = P / A$):
\[ j^\star = \sigma T^4 \]
That equation is Stefan's law, and $\sigma$ is the Stefan-Boltzmann constant.
\[ \sigma := \frac{2 \pi^5 k_B^4}{15 c^2 h^3} \]
That works only for blackbodies though -- if we want to generalize it to work for other materials, we can say
\[ j^\star = a \sigma T^4 \]
Where $a$ is the absorptivity/emissivity of the material (which is 1 for blackbodies, and somewhere between 0 and 1 for other materials).

\section{Debye sphere}
Stefan's law doesn't have a clear intuitive interpretation, but here's one way to think about it. Suppose you're in $k$-space, and have a sphere of radius $k_D$ that encloses pretty much all of the occupied $k$-states. According to the equipartition theorem, energy is proportional to $k_B T$, so energy per $k$-state is proportional to $T$. But according to the Planck relation and the equipartition theorem, the energy corresponding to a photon with wavenumber $k_D$ is
\[ E = h c k_D / 2 \pi = k_B T \]
which means as $T$ increases, the radius of the Debye sphere must increase proportionally to $T$, which means the number of occupied states enclosed by the sphere increases cubically with $T$. Since power is proportional to the number of states and also to the energy per state, it will increase quartically with temperature.

\end{document}
