\documentclass[12pt]{article}
\usepackage[margin=1in]{geometry}
\usepackage{amsmath}
\usepackage{amsfonts}
\usepackage{tikz-cd}

\begin{document}

\title{Matter Waves}
\author{Nathan Solomon}
\maketitle

\section{Cool Integral Trick}
Before starting the actual notes, here's a fun integral. It's particularly useful for solving all those problems where you average something over a probability distribution. It assumes $\operatorname{Re}(a)>0$ and $n \in \mathbb{N}$.
\begin{align*}
    \int_0^\infty x^n e^{-x/a} dx &= \int_0^\infty (a^n y^n) e^{-y} (a \, dy) \\
    &= a^{n+1} \int_0^\infty y^n e^{-y} dy \\
    &= a^{n+1} \Gamma(n+1) \\
    &= n! \, a^{n+1}
\end{align*}

\section{Another Unrelated Topic}
We should memorize this definition of the fine-structure constant:
\[ \alpha := \frac{e^2}{4 \pi \varepsilon_0 \hbar c} = \frac{e^2}{2 \varepsilon_0 h c} \approx \frac{1}{137} \]
It was also recommended that we remember the following approximations:
\[ \hbar c \approx 197 \text{ eV nm} \]
\[ \frac{e^2}{4 \pi \varepsilon_0} \approx 1.44 \text{ eV nm} \]

\section{Dispersion Relations}
This is pretty much all we need to know:
\[ v_\text{phase} = \omega / k \]
\[ v_\text{group} = \frac{\partial \omega}{\partial k} \]
For all matter waves, the geometric mean of the phase velocity and the group velocity is $c$, the speed of light. For non-dispersive waves, such as light traveling through a vacuum, the phase and group velocities are both $c$.

\section{Uncertainty Principle}

\section{Fourier Inversion Theorem}

\end{document}
