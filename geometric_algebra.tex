\documentclass[12pt]{article}
\usepackage[margin=1in]{geometry}
\usepackage{amsmath}
\usepackage{amsfonts}
\usepackage{tikz-cd}

\begin{document}

\title{Geometric Algebra}
\author{Nathan Solomon}
\maketitle

\section{General rules}
These should work no matter what the degree of $a$ and $b$ are. Note that the degree of a multivector is only defined if all its terms have the same degree.
\begin{itemize}
    \item Associativity: $(ab)c = a(bc)$
    \item Left distributivity: $a(b+c) = ab+ac$
    \item Right distributivity: $(a+b)c = ac+bc$
    \item The dot product and wedge product are also both left and right distributive
    \item Scalar multiplication (let $\lambda$ be a scalar): $\lambda a = a \lambda = \lambda \cdot a = a \cdot \lambda = \lambda \wedge a = a \wedge \lambda$
    \item Dot product commutativity: $a \cdot b = b \cdot a$
    \item Wedge product anticommutativity: $a \wedge b = (-1)^{\deg(a) \deg(b)} b \wedge a$
    \item $\deg(a \cdot b) = | \deg(a) - \deg(b) |$
    \item $\deg(a \wedge b) = \deg(a) + \deg(b)$ (if the degree of a multivector is greater than the dimension of the vector space it lives in, it must be zero. For example, any trivector in 2D will be written in terms of $e_1 \wedge e_1$ and $e_2 \wedge e_2$, which are both zero)
\end{itemize}

\section{Rules that work for blades (but not all multivectors)}
Terminology: a $k$-blade is a multivector that can be written as the wedge product of $k$ vectors. A multivector is any combination of blades -- for example, $e_1 e_2 + e_3 e_4 \in G(4,0)$ and $1 + e_1 \in G(1,0)$ are multivectors but not  blades.

\begin{itemize}
    \item Euclidean metric: $a^2 = ||a||^2$ where $||a||$ is a scalar
    \item Factorability: if $\deg(a) = x + y$, then $a$ can be written as the geometric product of an $x$-blade and a $y$-blade.
\end{itemize}

\section{Rules that might not work unless $a$ and $b$ are vectors}
\begin{itemize}
    \item $ab = a \cdot b + a \wedge b$
    \item $a \wedge b = - b \wedge a$
    \item $a \cdot b = \frac{1}{2}(ab + ba)$
    \item $a \wedge b = \frac{1}{2} (ab - ba)$
\end{itemize}

\section{Unit pseudoscalar $I$}
Let $I \in G(n, 0) = e_1 \wedge e_2 \wedge \cdots \wedge e_n$. Then $I$ commutes with everything. If $n \in (4 \mathbb{Z} + \{0, 1\})$, we have the very nice property that $I^{-1} = I$, meaning that taking the dual of something twice has no net effect. If $n \in (4 \mathbb{Z} + \{2, 3\})$, we have the elegant-but-not-as-nice property that $I^{-1} = -I$, meaning that the unit pseudoscalar $I$ behaves like the imaginary unit $i = \sqrt{-1}$. Using geometric algebras in mischievous ways, we can coerce them into representing complex numbers, or even quaternions.

Reversing a list of $n$ unique elements by swapping adjacent elements requires $n(n-1)/2$ swaps (you can prove that with induction). That number of swaps is even if $n \equiv 0 \mod{4}$ or $n \equiv 1 \mod{4}$, and odd otherwise. Thus, the permutation that reverses $e_1 \wedge e_2 \wedge \cdots \wedge e_n$ is even if $n \in (4 \mathbb{Z} + \{0, 1\})$ and odd otherwise, which is why that property above works.

\subsection{Cross product \& wedge product}
The cross product of vectors exists only in $\mathbb{R}^3$, so the following properties work only in $G(3, 0)$.
\[I (a \times b) = a \wedge b\]
\[a \times b = -I (a \wedge b) = I (b \wedge a)\]


\section{Cool application of combinations}
The set of multivectors of degree $p$ in $\mathbb{F}^n$ has dimension ${n \choose p} = \frac{n!}{p!(n-p)!} $ over $\mathbb{F}$. Thus, the set of all multivectors in $\mathbb{F}^n$ has dimension $2^n$. For example, $G(3,0)$ (the set of multivectors in $\mathbb{R}^3$) has the basis $\{1, e_1, e_2, e_3, e_1 e_2, e_2 e_3, e_3 e_1, e_1 e_2 e_3\}$ over $\mathbb{R}$, which matches the 1-3-3-1 row from Pascal's triangle.

\end{document}
