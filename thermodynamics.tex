\documentclass[12pt]{article}
\usepackage[margin=1in]{geometry}
\usepackage{amsmath}
\usepackage{amsfonts}
\usepackage{tikz-cd}

\begin{document}

\title{Thermodynamics}
\author{Nathan Solomon}
\maketitle

\section{Definitions of entropy \& temperature}
The entropy, $S$, of a system is defined in terms of the logarithm of the number of accessible microstates:
\[S := k_B \ln{\Omega(E)}\]

We can define temperature, $T$, in terms of the relationship between $E$ and $\Omega(E)$, if we pretend that $E$ and $\Omega(E)$ are differentiable.

\[\frac{1}{T} = \frac{\partial S}{\partial E} = \frac{k_B}{\Omega(E)} \cdot \frac{\partial \Omega(E)}{\partial E}\]

That simplifies to
\[T = \frac{\Omega(E)}{k_B \Omega'(E)}\]

TODO: COME UP WITH AN INTUITIVE WAY TO INTERPRET TEMPERATURE AS IT IS DEFINED HERE


\section{Boltzmann factor}
\subsection{Derived using partial derivatives}
\subsection{Derived using Stirling's approximation}


\section{Derivation of ideal gas law}

\section{Derivation of Maxwell-Boltzmann velocity distribution}
Find formula for median, mode, mean, and RMS

\section{Stern–Gerlach experiment}

\section{Other misc stuff from textbooks?}


\end{document}
