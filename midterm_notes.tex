\documentclass[12pt]{article}
\usepackage[margin=1in]{geometry}
\usepackage{amsmath}
\usepackage{amsfonts}
\usepackage{tikz-cd}

\begin{document}

\title{Midterm Notes}
\author{Nathan Solomon}
\maketitle

\section{Notecard for midterms \& final}
This document has all of the most important equations to include on your notecard EXCEPT for the stuff that was in notes from previous weeks (e.g. the Maxwell-Boltzmann speed distribution, the blackbody radiation equation, et cetera).

\section{Partition function $Z$}
\[ Z := \sum_i g_i e^{-\beta E_i} = \int g e^{- \beta E} dE \]
\[E_{avg} = \frac{\sum_i E_i e^{-\beta E_i}}{\sum_i e^{-\beta E_i}} = \frac{\int E e^{-\beta E} dE}{\int e^{-\beta E} dE}  = - \frac{\partial}{\partial \beta} \ln Z\]

\section{Spherical coords}
\begin{align*}
    x &= r \sin \theta \cos \phi \\
    y &= r \sin \theta \sin \phi \\
    z &= r \cos \theta \\
    dx \, dy \, dz &= r^2 \sin \theta \, dr \, d\theta \, d\phi
\end{align*}

\section{Summation identities}
The following Taylor series converges on $x \in (-1, 1)$.
\[\sum_{n=0}^\infty x^n = \frac{1}{1-x} \]
Differentiate that and multiply by $x$ to get this formula:
\[\sum_{n=0}^\infty nx^n = \frac{x}{(1-x)^2} \]

\section{Binomial coefficients}
\[ \binom{n}{k} := \frac{n!}{k! (n-k)!} \]
Pascal's identity:
\[\binom{n}{k} = \binom{n-1}{k} + \binom{n-1}{k-1}\]
Binomial theorem:
\[(a+b)^n = \sum_{k=0}^n \binom{n}{k} a^k b^{n-k}\]

\section{Energy-momentum equation}
In general, I'll use $m$ to denote the rest mass, and $\gamma m$ to mean the relativistic mass.
\[E^2 = m^2 c^4 + p^2 c^2\]
In that equation, $E$ is the kinetic energy plus the rest energy. The momentum is $p = \gamma m v$, and the rest energy is $mc^2$, so the kinetic energy must be
\begin{align*}
    K &= E - mc^2 \\
      &= \sqrt{m^2 c^4 + p^2 c^2} - mc^2 \\
      &= \sqrt{m^2 c^4 + \frac{m^2 v^2 c^2}{1 - \frac{v^2}{c^2}}} - mc^2 \\
      &= m c \sqrt{\frac{(c^2 - v^2) + v^2}{1 - \frac{v^2}{c^2}}} - mc^2 \\
      &= m c^2 \left( \sqrt{\frac{1}{1 - \frac{v^2}{c^2}}} - 1 \right) \\
      &= (\gamma - 1) m c^2
\end{align*}
This makes sense, because then the total energy is $E = \gamma m c^2$.

\section{Maxwell's equations}
Let $D := \varepsilon_0 E + P$ where $P$ is the polarization field, and let $H := \frac{B}{\mu_0} - M$ where $M$ is the magnetization field. Then the following version of Maxwell's equations work even if you aren't in a vacuum:
\begin{align*}
    \nabla \cdot D &= \rho \\
    \nabla \cdot B &= 0 \\
    \nabla \times E &= - \frac{\partial B}{\partial t} \\
    \nabla \times H &= J + \frac{\partial D}{\partial t}
\end{align*}
If you're in a vacuum, you can use the identity $\nabla \times (\nabla \times E) = \nabla (\nabla \cdot E) - \nabla^2 E$ to derive the 3D wave equation for electromagnetic waves:
\[ \nabla^2 E = \frac{1}{c^2} \cdot \frac{\partial^2 E}{\partial t^2} \]

\section{Maxwell-Boltzmann speed distribution}
\begin{align*}
    \text{probability density} &= 4 \pi v^2 \Big( \frac{m}{2 \pi k_B T} \Big)^{3/2} e^{-\frac{m v^2}{2 k_B T}} dv \\
                               &= 4 \sqrt{E/\pi} \Big( \frac{1}{k_B T} \Big)^{3/2} e^{-E / k_B T} dE
\end{align*}

mean speed: $\sqrt{ \frac{8kT}{\pi m} }$

RMS speed: $\sqrt{ \frac{3kT}{m} }$

mode speed: $\sqrt{ \frac{2kT}{m} }$

TODO: ADD MEAN, RMS, AND MODE ENERGIES TOO

\section{Wien's law}
\[ \lambda_\text{max} T(\lambda_\text{max}) = 2.898 \times 10^{-3} mK \]

\section{Angular frequency version of Fourier transform}
If $f$ is a function of $x$:
\[ (\mathcal{F}f)(k) = \frac{1}{\sqrt{2 \pi}} \int_{-\infty}^\infty e^{-ikx}f(x) dx \]
If $f$ is a function of $k$:
\[ (\mathcal{F}^{-1}f)(x) = \frac{1}{\sqrt{2 \pi}} \int_{-\infty}^\infty e^{ikx}f(k) dk \]

\section{Bohr model}
\begin{align*}
    r =& \frac{n^2 h^2 \varepsilon_0}{\pi m Z e^2} \approx& \frac{n^2}{Z}  (5.292\times 10^{-11} \text{ m}) \\
    v =& \frac{Z e^2}{2 n \varepsilon_0 h} \approx& \frac{Z}{n} (2.188\times 10^{6} \text{ m/s})
\end{align*}

\section{Constants}
\begin{align*}
    k_B &\approx 1.381 \times 10^{-23} \, \frac{\text{J}}{\text{K}} \approx 8.617 \times 10^{-5} \, \frac{\text{eV}}{\text{K}} \\
    h &\approx 6.626 \times 10^{-34} \text{ J s} \approx 4.136 \times 10^{-15} \text{ eV s} \\
\hbar := \frac{h}{2 \pi} &\approx 1.055 \times 10^{-34} \text{ J s} \approx 6.582 \times 10^{-16} \text{ eV s} \\
    c &\approx 2.998 \times 10^8 \, \frac{\text{m}}{\text{s}^2} \\
    k := \frac{1}{4 \pi \varepsilon_0} &\approx 8.988 \times 10^9 \, \frac{\text{N m}^2}{\text{C}^2} \\
    \varepsilon_0 &\approx 8.854 \times 10^{-12} \, \frac{\text{F}}{\text{m}} \\
    \mu_0 &\approx 1.257 \times 10^{-6} \, \frac{\text{N}}{\text{A}^2} \\
    1 \text{ mol} &\approx 6.022 \times 10^{23} \text{ molecules} \\
    \alpha := \frac{e^2}{2 \varepsilon_0 h c} &\approx 0.007297 \approx \frac{1}{137} \\
    a_0 = \frac{\varepsilon_0 h^2}{\pi e^2 m_e} &\approx 5.292 \times 10^{-11} \text{ m} \\
    G &\approx 6.674 \times 10^{-11} \, \frac{\text{N m}^2}{\text{kg}^2} \\
    e &\approx 1.602 \times 10^{-19} \text{ C} \\
    m_e &\approx 9.109 \times 10^{-31} \text{ kg} \approx 0.511 \text{ MeV/c}^2 \\
    m_p &\approx 1.673 \times 10^{-27} \text{ kg} \approx 1836 \, m_e \approx 938.3 \text{ MeV/c}^2 \\
    \sigma := \frac{2 \pi^5 k^4}{15 h^3 c^2} &\approx 5.670 \times 10^{-8} \, \frac{W}{m^2K^4} \\
    R &\approx \frac{25}{3} \, \frac{J}{mol \cdot K} \\
\end{align*}

\section{Other random stuff}
\begin{itemize}
    \item Average energy at 0 Kelvin is 3/5 of the Fermi energy (prove by integrating over spherical shells in k-space)
    \item Bragg scattering: $n \lambda = 2 d \sin \theta$
    \item $p=\hbar k= h/\lambda_\text{de Broglie}$
    \item $E = h f = \hbar \omega = \frac{h c}{\lambda} = \frac{\hbar c}{k}$
    \item $\lambda k = 2 \pi$
    \item If the work function $\phi$ is $e \cdot V_\text{stopping}$, then $KE_e = \frac{hc}{\lambda} - \phi$
    \item $v_\text{phase} = c \sqrt{1 + \left( \frac{mc}{\hbar k}  \right)^2}$
    \item $dN = (4 \pi k^2 dk) \left( \frac{L}{2 \pi} \right)^3 (\text{\# of polarization states})$
\end{itemize}

\end{document}
